\begin{figure}[htbp]
    \centering
    
    \begin{subfigure}[b]{0.48\textwidth}
        \centering
        \includegraphics[width=\linewidth]{rezultaty/warstwy/blednie_przewidziane_podpisy/Picture1.png}
        \caption{ID=318556}
        \label{fig:bledne_podpisy_a}
    \end{subfigure}
    \hfill 
    \begin{subfigure}[b]{0.48\textwidth}
        \centering
        \includegraphics[width=\linewidth]{rezultaty/warstwy/blednie_przewidziane_podpisy/Picture5.png}
        \caption{ID=203564}
        \label{fig:bledne_podpisy_b}
    \end{subfigure}

    \vspace{1em} 
    \begin{subfigure}[b]{0.48\textwidth}
        \centering
        \includegraphics[width=\linewidth]{rezultaty/warstwy/blednie_przewidziane_podpisy/Picture4.jpg}
        \caption{ID=391895}
        \label{fig:bledne_podpisy_c}
    \end{subfigure}
    \hfill
    \begin{subfigure}[b]{0.48\textwidth}
        \centering
        \adjincludegraphics[width=0.6\linewidth, trim={0 {.1\width} 0 {.02\width}}, clip]{rezultaty/warstwy/blednie_przewidziane_podpisy/Picture3.png}
        \caption{ID=324424}
        \label{fig:bledne_podpisy_d}
    \end{subfigure}

    \caption{Przykładowe obrazy, dla których model wygenerował błędne opisy. Typowe błędy obejmują (\textbf{a}) nieprawidłowe zliczenie obiektów, (\textbf{b}) pominięcie istotnych elementów sceny, (\textbf{c}) błędną identyfikację obiektu oraz (\textbf{d}) wygenerowanie podpisu, który jest zbyt ogólny. Szczegółowa analiza błędów znajduje się w Tabeli~\ref{tab:bledne_podpisy}.}
    \label{fig:bledne_podpisy_fig}
\end{figure}

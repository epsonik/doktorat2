\begin{table}
\caption{Rezultaty badań  z wykorzystaniem LSTM jako sieci rekurencyjnej oraz łączenia cech obrazu i tekstu metodą dodawania.}\label{tab:mergingADD}
\centering

\resizebox{\textwidth}{!}{
\begin{tabular}{|l|l|l|l|l|l|l|l|l|l|l|l|l}
\hline
& \textbf{\begin{tabular}[c]{@{}l@{}}Rozmiar \\ RNN\end{tabular}~}& \textbf{RKA} & \textbf{RKPS} & \textbf{p(mln)}  &\textbf{ t[ms]}& \textbf{B-1}& \textbf{B-2}& \textbf{B-3}& \textbf{B-4}& \textbf{C}& \textbf{S}\\ \hline
1   & 512      & 512                   & 512       &28,82&     5565           & 67,53   & 49,67   & 35,39   & 25,19   & 82,48 & 15,75 \\ \hline
2   & 256      & 256                   & -           &25,18&    4195       & 66,87   & 48,61   & 34,17   & 24,00     & 78,55 & 15,39 \\ \hline
3   & 128      & 128                   & 128           &23,7&   5100         & 65,86   & 47,94   & 33,54   & 23,40    & 74,92 & 14,77 \\ \hline
4   & 256      & 256                   & 256             &25,2&  6336        & 66,59   & 48,63   & 34,34   & 24,33     & 78,13 & 15,16 \\ \hline
\end{tabular}}
\end{table}
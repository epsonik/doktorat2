
\begin{figure}[htbp]
\begin{subfigure}{\textwidth} 
        \centering
        \includesvg[width=0.9\linewidth]{wykresy/scalanie_wstrzykiwanie/wstrzykiwanie_poczatkowe}
        \caption{Wstrzykiwanie inicjalne – cechy obrazu jako inicjalny stan ukryty RNN.}\label{fig:wstrzykiwanie_inicjalne}
    \end{subfigure}
    \vspace{1em}
    \begin{subfigure}{\textwidth}
        \centering
        \includesvg[width=0.9\linewidth]{wykresy/scalanie_wstrzykiwanie/wstrzykiwanie_wstepne}
        \caption{Wstrzykiwanie wstępne – obraz jako pierwsze słowo wejściowe do RNN.}
        \label{fig:wstrzykiwanie_wstepne}
    \end{subfigure}

    \vspace{1em}
    \begin{subfigure}{\textwidth}
        \centering
        \includesvg[width=0.75\linewidth]{wykresy/scalanie_wstrzykiwanie/wstrzykiwanie_rownolegle}
            \caption{Wstrzykiwanie równoległe – obraz i słowo jako wejście w każdym kroku czasowym.}
        \label{fig:wstrzykiwanie_rownolegle}
    \end{subfigure}

    \vspace{1em}
    \begin{subfigure}{\textwidth}
        \centering
        \includesvg[width=0.75\linewidth]{wykresy/scalanie_wstrzykiwanie/architektura_scalania}
            \caption{Architektura fuzji – połączenie wyjścia RNN i cech obrazu w warstwie multimodalnej.}
        \label{fig:architektura_scalania}
    \end{subfigure}

    \caption{Zestawienie popularnych architektur wstrzykiwania cech obrazu (a-c) oraz architektury fuzji (d)~\cite{TantiGattCamilleri2018Where}.}
    \label{fig:scalanie_wstrzykiwanie}
\end{figure}